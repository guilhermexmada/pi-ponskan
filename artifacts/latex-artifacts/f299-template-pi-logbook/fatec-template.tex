%%%% fatec-article.tex, 2024/03/10

%% Classe de documento
\documentclass[
landscape,
  a4paper,%% Tamanho de papel: a4paper, letterpaper (^), etc.
  12pt,%% Tamanho de fonte: 10pt (^), 11pt, 12pt, etc.
  english,%% Idioma secundário (penúltimo) (>)
  brazilian,%% Idioma primário (último) (>)
]{article}

%% Pacotes utilizados
\usepackage[]{fatec-article}
\usepackage{setspace}

%% Processamento de entradas (itens) do índice remissivo (makeindex)
%\makeindex%

%% Arquivo(s) de referências
%\addbibresource{fatec-article.bib}

%% Início do documento
\begin{document}

% Seções e subseções
%\section{Título de Seção Primária}%

%\subsection{Título de Seção Secundária}%

%\subsubsection{Título de Seção Terciária}%

%\paragraph{Título de seção quaternária}%

%\subparagraph{Título de seção quinária}%

%\section*{Diário de Bordo}%
\section*{Instruções para o preenchimento}
\doublespacing
\begin{enumerate}
    \item O Diário de Bordo é usado para registrar atividades, progressos, ideias e desafios enfrentados em um projeto ou durante a rotina de trabalho. Serve como um registro cronológico e detalhado das operações diárias, facilitando a organização e o acompanhamento das tarefas.
    \doublespacing
    \item Durante o registro das atividades deve-se incluir detalhes como datas, horários, descrições de tarefas, nomes de participantes e observações relevantes.  Esta documentação contínua ajuda na avaliação do progresso de projetos ou atividades, permitindo ajustes e melhorias contínuas nos processos.
    \doublespacing
    \item Para evidenciar a realização das tarefas, você poderá utilizar a criação de anexos para adicionar anotações, fotos, prints, questionários, entre outros.
\end{enumerate}

\break

 \begin{table}[]
\centering
\begin{tabular}{|l|l|l|l|l|}
\hline
Nome da Atividade & Data de início & Data de término & Responsável pela atividade & Descrição da atividade realizada \\ \hline
Forproex                           &  05/09/2025                & 05/09/2025     & Todos os membros& Definição de área temática e linhas de pesquisa                                  \\ \hline
Canvas Business Model              &  10/09/2025                & 12/09/2025     & Gustavo         & Criação do modelo de negócio                                 \\ \hline
Identidade Visual                  &                            &                & Matheus         &                                  \\ \hline
Artigo Científico                  &  24/09/2025                & 06/10/2025     & Guilherme       & Escrita da Introdução, Objetivo e Estado da Arte em Latex                                 \\ \hline
Landing Page                       &  20/09/2025                & 08/10/2025     & João e Guilherme& Atualização da landing page com novo PI                                 \\ \hline
Diagrama de redes                  &                            &                & Gustavo         &                                  \\ \hline
Protótipo de alta fidelidade       &                            &                & Matheus         &                                  \\ \hline
Diagrama de caso de uso            &  10/09/2025                & 15/09/2025     & Guilherme       & Diagramação dos casos de uso do sistema no Draw.io                                 \\ \hline
Banner                             &                            &                & Guilherme       &                                  \\ \hline
Diagramas de banco de dados        &  25/09/2025                & 10/10/2025     & Arthur          & Modelagem conceitual e lógica no Brmodelo                                 \\ \hline
Pitch com IA                       &                            &                & Gustavo         &                                  \\ \hline
Front-end Web                      &  10/10/2025                & 23/10/2025     & Matheus         & Codificação das interfaces do sistema web                                 \\ \hline
Back-end Web                       &  10/10/2025                & 27/10/2025     & Matheus         & Codificação das operações CRUD no sistema web                                 \\ \hline
Banco de dados físico              &  18/09/2025                & 16/09/2025     & Arthur          & Codificação do banco de dados relacional                                  \\ \hline
Diagrama de classes                & 09/10/2025                 & 14/10/2025     & Guilherme       & Diagramação das classes principais no Draw.io                                 \\ \hline
Diagrama de objetos                & 12/10/2025                 & 14/10/2025     & Guilherme       & Diagramação dos objetos de exemplo no Draw.io                                 \\ \hline
Diário de bordo                    & 10/09/2025                 & 28/10/2025     & Guilherme       & Documentação das tarefas do PI                                 \\ \hline

\end{tabular}
\end{table}



\end{document}