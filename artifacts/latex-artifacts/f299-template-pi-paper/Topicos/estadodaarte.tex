Segundo Molin, Amaral e Colaço (2015), a agricultura de precisão (AP) é uma abordagem da agricultura que pode ser definida de várias formas, a depender do ponto de vista analisado. Em sua fase inicial, estava fortemente vinculada ao georreferenciamento (GPS) e, mais recentemente, pode ser entendida como a aplicação de Tecnologia da Informação (TI) durante a condução das lavouras. De qualquer maneira, a AP está fundamentada na compreensão de que as propriedades agrícolas não são uniformes espacialmente ou ao longo do tempo, o que exige o desenvolvimento de estratégias para gerenciar a heterogeneidade das lavouras e seus respectivos problemas.

Nesse contexto, o mercado nacional da AP já apresenta muitas soluções avançadas e diferentes, dentre as quais se destacam a ampla adoção de drones, tratores autônomos, sensores e serviços de monitoramento. Segundo a \textcite{ANAC2025}, entre 2023 e 2024 o número de cadastros de drones agrícolas saltou de 674 para 7.312, o que configura um aumento de mais de 10 vezes. Sensores multiespectrais, climáticos e de solo ajudam os produtores a acompanhar e ajustar processos como adubação e irrigação com bases de dados precisos, e quando integrados a plataformas digitais, entende-se como uso da Internet das Coisas (IOT) para reduzir desperdícios ao permitir monitoramento remoto das fazendas. Em relação a plataformas digitais para compra e venda, uma pesquisa realizada em 2020 pelo conjunto Embrapa, Sebrae e Inpe revelou que são utilizadas por pelo menos 40\% dos produtores. 

Nesse retrato,  a visão computacional “pode ser empregada na detecção de doenças e pragas, na estimação de safra e na avaliação não invasiva de atributos como qualidade, aparência e volume, além de ser componente essencial em sistemas robóticos agrícolas” \cite{santos2020visao}. O mesmo estudo aborda a classificação de padrões associados a objetos de interesse e sintomas de doenças e pragas como problemas  perceptuais abordados pela visão computacional, geralmente capturados em imagens por câmeras acopladas em tratores, robôs e drones. Cita também os problemas geométricos, advindos da perda de informações de profundidade em imagens bidimensionais e que podem ser recuperadas por algoritmos de visão computacional que interpretem múltiplas imagens de uma mesma cena.

Uma das soluções para detecção de doenças baseada na visão computacional e dedicada à citricultura é o trabalho de \cite{Pazoti2005}, que propôs o CITRUSVIS como sistema para auxiliar na identificação e contagem automática de ascósporos do fungo G. citricarpa, causador da pinta preta, a fim de reduzir o tempo e os erros associados à análise manual. O fluxo de funcionamento inicia com a aquisição de imagens digitais de discos de coletas - são materiais acrílicos e adesivos onde se depositam os ascósporos e outras partículas sólidas sugadas por um equipamento chamado caça-esporo. Os discos são tingidos de corante láctico azul para melhor visualização e fotografados por microscópio por uma câmera digital de 640 x 512 pixels. O pré-processamento das imagens incluiu a conversão para padrão HSI (matiz, saturação e intensidade) para contrastar e separar esporos corados do fundo branco; técnica de limiarização (thresholding) que define cada pixel como preto ou branco; filtros não lineares e operações morfológicas que suavizam os ruídos na imagem preservando as bordas dos esporos e a técnica transformada watershed, que isola esporos colados ou sobrepostos. Após a extração das características de curvatura da forma e descritores de Fourier, a classificação usou um modelo de rede neural artificial (RNA) com aprendizado supervisionado, o qual recebeu 60 descritores de entrada que passaram por 2 camadas escondidas, com 20 e 15 neurônios. Como alternativa, testaram o classificador por método da distância mínima, que atribui a partícula à classe pelo vetor de características médias.

O protótipo foi integrado com uma interface gráfica (GUI) no software de cálculo numérico MATLAB. Com relação aos resultados, a RNA obteve 98\% de acurácia em amostras pré-selecionadas e 96,6\% em testes gerais, superando os 92\% do classificador por distância mínima; além disso, o tempo médio de processamento para cada imagem foi de 66 segundos. Concluiu-se um desempenho satisfatório na identificação dos ascósporos, podendo substituir parcialmente a análise manual dos especialistas, apesar de limitações na qualidade das imagens oriundas da fotografia em microscópio, como ruídos, sobreposição e baixa resolução.

Outro trabalho que utilizou redes neurais artificiais para processar imagens foi o de \cite{Ribeiro2012} que objetivou a diferenciação do greening de outras doenças foliares e deficiências nutricionais em citros, uma vez que a identificação visual é falha e a Reação em Cadeia de Polimerase (PCR), técnica molecular para identificar DNA/RNA de organismos com alta precisão, é cara e demorada. Foram usadas amostras de 60 folhas de citros com sintomas de greening, CVC, rubelose e deficiências de magnésio, manganês e zinco, cujas imagens foram digitalizadas em scanner fotográfico. O pré-processamento baseou-se na segmentação das cores verdes, amarelas e marrons das manchas do greening através de uma RNA treinada com 46 imagens, e as restantes de validação. Todo o treinamento foi feito por pixels rotulados, que passaram por 3 camadas escondidas e ganharam máscaras binárias que definem as manchas. Foram extraídos descritores de forma ao aplicar cálculos matemáticos nas máscaras binárias, gerando uma quantidade de atributos que serão guardados em vetores e, depois, em quadrantes que medem a assimetria das manchas. A classificação das doenças também usou RNA e construiu escalas de severidade dos sintomas em diagrama. 

O estudo comentou desafios como a alta similaridade dos sintomas, a amostra reduzida de 60 folhas e o uso isolado da cor amarela, que não foi suficiente para discriminar o greening. Ainda sim, a segmentação por cor com RNA obteve 96,04\% de precisão. No entanto, na classificação geral, considerando todas as doenças, os resultados variaram entre 43\% e 63\%. Concluiu-se que a RNA foi viável para auxiliar a diferenciação do greening e pode ser usada para complementar a inspeção visual no campo, porém é necessário combinar a classificação de cores com os descritores de forma para alavancar a acurácia.

Apesar dos dois artigos possuírem métodos de captura de imagens complexos ou demasiadamente manuais, ambos evidenciam o potencial das redes neurais artificiais e técnicas de processamento de cor na detecção e classificação de manchas associadas a doenças. 

Um estudo mais próximo dos objetivos deste projeto e também recente foi o de \textcite{Momeny2022}, que se aprofundou no desenvolvimento de redes neurais convolucionais (CNNs) aliadas a uma estratégia chamada “learning-to-augment”, que cria dados artificiais para treinamento do modelo na detecção da pinta preta (conhecida em inglês como Black Spot) em laranjas. A base de dados (dataset) utilizada totalizou 1896 imagens de laranjas divididas em classes de frutos verdes, meia-maturação, maduros e com pinta preta - todas capturadas com smartphone Samsung SM-J500H com câmera de 4128 x 2322 pixels. Também foi utilizada uma versão do MATLAB como software de processamento, e alguns modelos de deep learning foram testados, como MobileNetV2, GoogleNet e DenseNet201. Os resultados demonstraram melhor desempenho do modelo ResNet50, que alcançou acurácia total de 99,5\% com 100\% de sensibilidade e especificidade, ou seja, sem desperceber frutos doentes nem confundir frutos saudáveis. 

\textcite{Momeny2022} mostrou que um sistema rápido e de baixo custo, com câmeras de smartphones, pode ser aplicado para detecção da pinta preta. Além disso, obteve êxito no uso do “learning-to-augment” sobre técnicas tradicionais de aumento de amostras de dados, como rotação e flip. Considerando que as CNNs são tipos específicos de RNAs especializadas em dados com estrutura espacial, como imagens, áudios e vídeo, este projeto oferece uma dica para o desenvolvimento da rede neural para classificação das imagens da pinta preta na tangerina ponkan. 