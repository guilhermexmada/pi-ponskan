Os Objetivos de Desenvolvimento Sustentável (ODSs) foram adotados pela Organização das Nações Unidas \cite{ONU2025} no ano de 2015, estabelecendo parâmetros para políticas nacionais sobre 17 temáticas que impactam o desenvolvimento humano em todos os países, que devem alcançá-los até o ano de 2030, por meio da cooperação internacional. Entre esses objetivos, o projeto de identificação da pinta preta (\emph{Phyllosticta citricarpa}) por visão computacional relaciona-se profundamente com as ODSs 2, 9 e 12, que propõem medidas para garantir segurança alimentar com agricultura sustentável, fomento da inovação nos setores produtivos e eficiência no uso dos recursos naturais, respectivamente. Esses objetivos serão trabalhados com enfoque na tangerina ponkan (\emph{Citrus reticulata}), uma cultura de destaque na região do Vale do Ribeira que pode contrair uma diversidade de doenças e avarias com sintomas semelhantes aos da pinta preta.

Sendo o Brasil o maior produtor mundial de laranja e suco de laranja \cite{USDA2025}, torna-se evidente a importância nacional deste gênero de frutas, conhecidas como citros (\emph{Citrus}). Fazem parte deste gênero as espécies de laranja, limão, lima, tangerina/mexerica, pomelo/toranja e cidra. Dentre elas, a tangerina possui um grande valor socioeconômico no Vale do Ribeira, uma área histórica e com forte agricultura familiar localizada entre os estados de São Paulo e Paraná. Um levantamento da Secretaria de Agricultura e Abastecimento (SEAB) realizado em 2020 revelou que o Vale produziu mais de 98 mil toneladas de citros, das quais 94\% foram tangerinas e 92\% destas foram da variedade ponkan. Somente as tangerinas renderam um valor bruto de produção de aproximadamente 168,3 milhões de reais para toda a citricultura da região, sendo a parte paranaense responsável por cerca de 163,2 milhões de reais \cite{Parana2021}.

Como dito anteriormente, a agricultura familiar é um pilar fundamental na economia do Vale do Ribeira, uma vez que contribui para a geração de renda, segurança alimentar e preservação ambiental em municípios onde os índices de desenvolvimento humano (IDH) geralmente estão abaixo da média estadual. Nesse contexto, a tangerina ponkan se adaptou bem às condições locais, pois a drenagem do solo incentiva o crescimento de raízes profundas para melhor captação de água; já o clima subtropical úmido, com alta umidade relativa do ar e temperaturas moderadas a altas, favorece a suculência e o tamanho dos frutos \cite{Borges2021}.

Essas características de agrado do consumidor definem a tangerina como um produto de comercialização principalmente \emph{in-natura}, abastecendo não apenas o mercado externo, mas toda uma cadeia de pequenas propriedades familiares que são afetadas diretamente pelos problemas da cultura. Por exemplo, a sazonalidade do mercado, em razão da safra concentrada entre os meses de abril e julho, pode afetar diretamente famílias que têm o cultivo de ponkan como fonte de renda principal. Outro fator decisivo é a produtividade, muitas vezes prejudicada por questões fitossanitárias, ou seja, pragas e doenças. 

O cenário nacional de citros enfrenta grandes ameaças de agentes nocivos como insetos, bactérias e fungos. Atualmente, o \emph{huanglongbing} ou \emph{greening} (HLB) é considerado a maior ameaça à cadeia citrícola, tendo em vista suas capacidades destrutivas e a ausência de medidas curativas \cite{rodrigues2016hlb}. A clorose variegada dos citros (CVC) é outra doença bacteriana relevante, a qual coloniza os vasos do xilema da planta e dificulta o transporte de nutrientes e água, reduzindo drasticamente a produtividade da lavoura \cite{Alves2003}. Já a pinta preta, segundo \textcite{SilvaJunior2016}, foi relatada em 1980 no estado do Rio de Janeiro e encontra-se praticamente em todas as regiões citrícolas do país. Trata-se de uma doença fúngica causada pelas variações de reprodução sexuada \emph{Phyllostictina citricarpa} e assexuada \emph{Guignardia citricarpa}. Caracteriza-se pela queda prematura dos frutos, cujas cascas são acometidas por lesões que também desqualificam significativamente o produto para comercialização. Por esse motivo, “o manejo da doença em pomares para o mercado interno deve ser muito rigoroso e eficiente” \cite{SilvaJunior2016}.

No entanto, o mesmo estudo aponta que os sintomas da pinta preta, principalmente as lesões, podem gerar confusão para o produtor. Isso acontece porque elas são dotadas de grande variabilidade e geralmente se assemelham a danos mecânicos ou insetos, com grandes chances da doença não ser detectada até que atinja com severidade a planta. Os principais sintomas são manchas dura, sardenta, virulenta, rendilhada, trincada e a falsa melanose, com manifestação favorecida pela exposição ao sol em altas temperaturas \cite{SilvaJunior2016}.

Quando observa-se a falta de treinamentos técnicos e a escassez tecnológica como realidade dominante na agricultura familiar, é possível inferir a ameaça que a pinta preta representa para a produção de tangerina ponkan no Vale do Ribeira. Outro agravante é o próprio clima local, que propicia as contaminações por fungos em razão das altas taxas de umidade e calor. Portanto, as medidas de prevenção também precisam ser desenvolvidas, visando evitar a disseminação do fungo pelos pomares e, consequentemente, diminuir os custos de controle. Dentre as práticas preventivas, destacam-se a inspeção da doença, controle de entrada de materiais e mudas certificadas, nutrição e sanidade do pomar e remoção dos frutos infectados antes da florada \cite{Fundecitrus2025}.

Para o propósito de otimizar os processos de prevenção da pinta preta na cultura da ponkan, este projeto pretende desenvolver uma ferramenta baseada em Visão Computacional, que decorre da evolução tecno-científica e, mais especificamente, da evolução dos algoritmos de Inteligência Artificial (IA). 

IA é compreendida como campo de estudo dos sistemas capazes de realizar tarefas propriamente humanas, nascida no período da Segunda Guerra Mundial e oficializada em 1956, com a reunião de cientistas na Conferência de \emph{Dartmouth} \cite{coelho2012turing}. Desde então, a IA cresceu ao ponto de, atualmente, automatizar e integrar soluções em diversos setores da sociedade, desde geradores de texto e imagens até carros, robôs e drones autônomos. Já a Visão Computacional é uma área da IA que procura emular o funcionamento da visão humana a partir do processamento de imagens, segundo \textcite{marengoni2009opencv}. A partir de bases de dados satisfatórias, algoritmos de Visão Computacional podem ser alimentados com vídeos e imagens e treinados para reconhecer formas, objetos, padrões de cores e texturas.

Este projeto utilizará esses princípios para produzir um sistema de identificação da pinta preta para telefones celulares, de forma a servir como uma ferramenta fácil e rápida de pré-diagnóstico para produtor, que poderá tomar decisões mais assertivas e inserir agricultura de precisão no seu manejo.