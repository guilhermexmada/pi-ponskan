\documentclass{beamer}
\usepackage[utf8]{inputenc}
\usepackage{lmodern}
\usepackage[brazil]{babel}
\usetheme{Madrid}

% -------------------------------------------------
% Identidade Visual - Centro Paula Souza
% -------------------------------------------------
\definecolor{cpsred}{RGB}{153,0,0}      % Vermelho institucional
\definecolor{cpsgray}{RGB}{85,85,85}    % Cinza técnico

\usecolortheme[named=cpsred]{structure}
\setbeamercolor{title}{fg=white,bg=cpsred}
\setbeamercolor{frametitle}{fg=white,bg=cpsred}
\setbeamercolor{structure}{fg=cpsred}
\setbeamercolor{normal text}{fg=black,bg=white}
\setbeamercolor{itemize item}{fg=cpsred}

\usepackage{ragged2e}
\usepackage{hyperref}

\setbeamertemplate{headline}{} 

% -------------------------------------------------
% Rodapé padrão
% -------------------------------------------------
\setbeamertemplate{footline}[frame number]
\addtobeamertemplate{footline}{
  \hfill\usebeamercolor[fg]{frametitle}{\hspace{-1.5cm}\scriptsize Prof. MSc. Frederico Barbosa Muniz}\hspace{1cm}
}{}

% -------------------------------------------------
% Informações principais
% -------------------------------------------------
\title[Workshop de Artigos Científicos]{\textbf{Workshop de Artigos Científicos}}
\author{Prof. MSc. Frederico Barbosa Muniz}
\date{}

% -------------------------------------------------
% Documento
% -------------------------------------------------
\begin{document}

% Capa
\begin{frame}
    \centering
    \vspace{1cm}
    {\color{cpsred}\Huge\textbf{Workshop de Artigos Científicos}}\\[0.8cm]
    {\Large Prof. MSc. Frederico Barbosa Muniz}\\[0.4cm]
    \textcolor{cpsgray}{Centro Paula Souza}\\[0.2cm]
    \textcolor{cpsgray}{FATEC Registro}
\end{frame}

% Sumário
\begin{frame}{Sumário}
\tableofcontents
\end{frame}

% -------------------------------------------------
% SEÇÕES E SLIDES
% -------------------------------------------------

\section{Estado da Arte inadequado}
\begin{frame}{Estado da Arte inadequado}
\justifying
Frequentemente não cumpre a função acadêmica: não discute trabalhos correlatos de computação, nem métodos, resultados e limitações.  
Descreve panoramas gerais ou plataformas, mistura explicações conceituais com revisão e, em vários casos, precisa ser totalmente refeito.
\end{frame}

\section{Referências e normas (ABNT) deficientes}
\begin{frame}{Referências e normas (ABNT) deficientes}
\justifying
Falta de padronização (itálico de periódicos, DOI, formatação).  
Incoerência entre o que é citado no texto e o que aparece na lista.  
URLs soltas, entradas incompletas e referências fora da norma.
\end{frame}

\section{Uso de fontes fracas ou inadequadas}
\begin{frame}{Uso de fontes fracas ou inadequadas}
\justifying
Citações provenientes de páginas não científicas, ausência de fontes oficiais, artigos revisados por pares ou documentos institucionais confiáveis.
\end{frame}

\section{Introduções mal estruturadas}
\begin{frame}{Introduções mal estruturadas}
\justifying
Ausência de dados quantitativos e de referências.  
Organização confusa e, em muitos casos, a tecnologia é apresentada antes da definição clara do problema.
\end{frame}

\section{Objetivos frágeis}
\begin{frame}{Objetivos frágeis}
\justifying
Objetivos gerais pouco operacionais, específicos ausentes ou mal apresentados.  
Há redundâncias e inversão entre o que é produto e o que é método.
\end{frame}

\section{Metodologia confusa ou incompleta}
\begin{frame}{Metodologia confusa ou incompleta}
\justifying
Mistura de etapas, ferramentas e resultados.  
Falta de clareza sobre o fluxo de desenvolvimento, separação entre metodologia de pesquisa e de implementação, e presença de trechos do modelo sem conteúdo autoral.
\end{frame}

\section{Elementos gráficos e estruturais problemáticos}
\begin{frame}{Elementos gráficos e estruturais problemáticos}
\justifying
Figuras muito pequenas ou sem legenda.  
Ausência de fluxogramas obrigatórios.  
Tabelas e seções de exemplo do modelo não substituídas adequadamente.
\end{frame}

\section{Citações mal posicionadas ou incompletas}
\begin{frame}{Citações mal posicionadas ou incompletas}
\justifying
Pedidos de “citação ao final do parágrafo” ignorados.  
Falta de identificação clara da fonte de dados e lacunas entre citações e lista de referências.
\end{frame}

\section{Linguagem e estilo}
\begin{frame}{Linguagem e estilo}
\justifying
Uso de tom informal, anglicismos sem formatação, excesso de parênteses, jargões não definidos e repetições de siglas e conceitos.
\end{frame}

\section{Resultados e discussão insuficientes}
\begin{frame}{Resultados e discussão insuficientes}
\justifying
Seções ausentes, apenas marcadores de “RESULTADOS” sem conteúdo analítico.  
Falta de métricas, comparações e análises quantitativas ou qualitativas.
\end{frame}

\section{Conclusões genéricas ou não redigidas}
\begin{frame}{Conclusões genéricas ou não redigidas}
\justifying
Presença de texto-modelo (“Apresente aqui as conclusões...”) e ausência de fechamento coerente com os objetivos propostos.
\end{frame}

\section{Inconsistências terminológicas e redundâncias}
\begin{frame}{Inconsistências terminológicas e redundâncias}
\justifying
Repetição de definições já apresentadas (como STRIDE) e reintrodução de conceitos sem aprofundamento teórico ou contextual.
\end{frame}

\section{Erros de revisão}
\begin{frame}{Erros de revisão}
\justifying
Problemas de ortografia, gramática e coesão textual, incluindo o uso incorreto de termos (“trás” em vez de “traz”) e falhas de concordância.
\end{frame}

\section{Questões éticas e metodológicas}
\begin{frame}{Questões éticas e metodológicas}
\justifying
Identificação nominal de profissionais ou participantes em seções de metodologia, o que compromete a ética e a confidencialidade dos dados coletados.
\end{frame}

\section{Palavras-chave e abstract inconsistentes}
\begin{frame}{Palavras-chave e abstract inconsistentes}
\justifying
Presença de blocos em inglês sem alinhamento com o texto.  
Erros de correspondência entre termos e ausência de revisão das traduções.
\end{frame}

\section{Encerramento}
\begin{frame}{Encerramento}
\justifying
Os problemas levantados indicam a necessidade de fortalecer a formação científica, a escrita acadêmica e o domínio metodológico dos alunos.  
O aprimoramento contínuo desses aspectos é essencial para a qualidade e a credibilidade dos projetos de pesquisa desenvolvidos na FATEC Registro.
\end{frame}

\end{document}
