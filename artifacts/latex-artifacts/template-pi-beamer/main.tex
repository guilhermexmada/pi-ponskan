\documentclass{beamer}
\usepackage[utf8]{inputenc}
\usepackage{lmodern}
\usepackage{graphicx}
\usepackage{wrapfig}
\usepackage{subcaption}
\usepackage[export]{adjustbox}
\usepackage[brazilian]{babel}
\usetheme{Madrid}

% -------------------------------------------------
% Identidade Visual - Centro Paula Souza
% -------------------------------------------------
\definecolor{cpsred}{RGB}{153,0,0}      % Vermelho institucional
\definecolor{cpsgray}{RGB}{85,85,85}    % Cinza técnico

\usecolortheme[named=cpsred]{structure}
\setbeamercolor{title}{fg=white,bg=cpsred}
\setbeamercolor{frametitle}{fg=white,bg=cpsred}
\setbeamercolor{structure}{fg=cpsred}
\setbeamercolor{normal text}{fg=black,bg=white}
\setbeamercolor{itemize item}{fg=cpsred}

\usepackage{ragged2e}
\usepackage{hyperref}

\setbeamertemplate{headline}{} 

% -------------------------------------------------
% Rodapé padrão
% -------------------------------------------------
\setbeamertemplate{footline}[frame number]
\addtobeamertemplate{footline}{
  \hfill\usebeamercolor[fg]{frametitle}{\hspace{-1.5cm}\scriptsize Prof. MSc. Frederico Barbosa Muniz}\hspace{1cm}
}{}

% -------------------------------------------------
% Informações principais
% -------------------------------------------------
\title[Sistema para Identificação da Pinta Preta na Tangerina Ponkan]{\textbf{Sistema para Identificação da Pinta Preta na Tangerina Ponkan}}
\author{Arthur Parra da Silva \and Guilherme Shimada Pereira \and Gustavo Kletelinger \and Matheus Bertoldo de Oliveira}
\date{}

% -------------------------------------------------
% Documento
% -------------------------------------------------
\begin{document}

% Capa
\begin{frame}
    \centering
    \vspace{1cm}
    {\color{cpsred}\Huge\textbf{Sistema para Identificação da Pinta Preta na Tangerina Ponkan}}\\[0.7cm]
    {\Large Arthur Parra da Silva }\\[0.2cm]
    {\Large Guilherme Shimada Pereira }\\[0.2cm]
    {\Large Gustavo Kletelinger }\\[0.2cm]
    {\Large Matheus Bertoldo de Oliveira }\\[0.4cm]
    \textcolor{cpsgray}{Centro Paula Souza}\\[0.2cm]
    \textcolor{cpsgray}{FATEC Registro}
\end{frame}

% Sumário
\begin{frame}{Sumário}
\tableofcontents
\end{frame}

% -------------------------------------------------
% SEÇÕES E SLIDES
% -------------------------------------------------

\section{Logomarca}
\begin{frame}{Logomarca}
\justifying

\begin{figure}[h]
\includegraphics[width=0.8\textwidth]{images/logomarca.png}
\caption{Logomarca}
\label{fig:figure2}
\end{figure}


As formas arredondadas lembram pontos conectados, fazendo referência às redes neurais, 
estruturas fundamentais para o projeto. O laranja é amigável e traz confiança, enquanto o verde
remete à crescimento, saúde e natureza.

\end{frame}

\begin{frame}{Logomarca}
\justifying

\begin{wrapfigure}{l}{0.25\textwidth}
\includegraphics[width=0.21\textwidth]{images/logotipo.png}
\caption{Logotipo}
\label{fig:wrapfig}
\end{wrapfigure}

As formas arredondadas lembram pontos conectados, fazendo referência às redes neurais, 
estruturas fundamentais para o projeto.


\end{frame}

\section{Agenda}
\begin{frame}{Referências e normas (ABNT) deficientes}
\justifying
Falta de padronização (itálico de periódicos, DOI, formatação).  
Incoerência entre o que é citado no texto e o que aparece na lista.  
URLs soltas, entradas incompletas e referências fora da norma.
\end{frame}

\section{Pitch}
\begin{frame}{Uso de fontes fracas ou inadequadas}
\justifying
Citações provenientes de páginas não científicas, ausência de fontes oficiais, artigos revisados por pares ou documentos institucionais confiáveis.
\end{frame}

\section{Problematização}
\begin{frame}{Introduções mal estruturadas}
\justifying
Ausência de dados quantitativos e de referências.  
Organização confusa e, em muitos casos, a tecnologia é apresentada antes da definição clara do problema.
\end{frame}

\section{Estado da Arte}
\begin{frame}{Objetivos frágeis}
\justifying
Objetivos gerais pouco operacionais, específicos ausentes ou mal apresentados.  
Há redundâncias e inversão entre o que é produto e o que é método.
\end{frame}

\section{Objetivo}
\begin{frame}{Metodologia confusa ou incompleta}
\justifying
Mistura de etapas, ferramentas e resultados.  
Falta de clareza sobre o fluxo de desenvolvimento, separação entre metodologia de pesquisa e de implementação, e presença de trechos do modelo sem conteúdo autoral.
\end{frame}

\section{Metodologia}
\begin{frame}{Elementos gráficos e estruturais problemáticos}
\justifying
Figuras muito pequenas ou sem legenda.  
Ausência de fluxogramas obrigatórios.  
Tabelas e seções de exemplo do modelo não substituídas adequadamente.
\end{frame}

\section{Apresentação Prática}
\begin{frame}{Citações mal posicionadas ou incompletas}
\justifying
Pedidos de “citação ao final do parágrafo” ignorados.  
Falta de identificação clara da fonte de dados e lacunas entre citações e lista de referências.
\end{frame}


\section{Encerramento}
\begin{frame}{Encerramento}
\justifying
Os problemas levantados indicam a necessidade de fortalecer a formação científica, a escrita acadêmica e o domínio metodológico dos alunos.  
O aprimoramento contínuo desses aspectos é essencial para a qualidade e a credibilidade dos projetos de pesquisa desenvolvidos na FATEC Registro.
\end{frame}

\end{document}
