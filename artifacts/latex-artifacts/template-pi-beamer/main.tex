\documentclass{beamer}
\usepackage[utf8]{inputenc}
\usepackage{lmodern}
\usepackage{graphicx} % imagens
\usepackage{wrapfig} % flutuação de imagens na borda do texto
\usepackage{subcaption} 
\usepackage[export]{adjustbox} % orientação de elementos
\usepackage{hyperref} % inserção de links
\usepackage{xcolor} % cores
\usepackage{tikz} % diagramas
\usepackage{multicol} % múltiplas colunas
\usepackage{blindtext}

\usepackage[brazilian]{babel}
\usetheme{Madrid}

% -------------------------------------------------
% Identidade Visual - Centro Paula Souza
% -------------------------------------------------
\definecolor{cpsred}{RGB}{153,0,0}      % Vermelho institucional
\definecolor{cpsgray}{RGB}{85,85,85}    % Cinza técnico

\usecolortheme[named=cpsred]{structure}
\setbeamercolor{title}{fg=white,bg=cpsred}
\setbeamercolor{frametitle}{fg=white,bg=cpsred}
\setbeamercolor{structure}{fg=cpsred}
\setbeamercolor{normal text}{fg=black,bg=white}
\setbeamercolor{itemize item}{fg=cpsred}

\usepackage{ragged2e}
\usepackage{hyperref}

\setbeamertemplate{headline}{} 

% -------------------------------------------------
% Rodapé padrão
% -------------------------------------------------
\setbeamertemplate{footline}[frame number]
\addtobeamertemplate{footline}{
  \hfill\usebeamercolor[fg]{frametitle}{\hspace{-1.5cm}\scriptsize Prof. MSc. Frederico Barbosa Muniz}\hspace{1cm}
}{}

% -------------------------------------------------
% Informações principais
% -------------------------------------------------
\title[Sistema para Identificação da Pinta Preta na Tangerina Ponkan]{\textbf{Sistema para Identificação da Pinta Preta na Tangerina Ponkan}}
\author{Arthur Parra da Silva \and Guilherme Shimada Pereira \and Gustavo Kletelinger \and Matheus Bertoldo de Oliveira}
\date{}

% -------------------------------------------------
% Documento
% -------------------------------------------------
\begin{document}

% Capa
\begin{frame}
    \centering
    \vspace{1cm}
    {\color{cpsred}\Huge\textbf{Sistema para Identificação da Pinta Preta na Tangerina Ponkan}}\\[0.7cm]
    {\Large Arthur Parra da Silva }\\[0.2cm]
    {\Large Guilherme Shimada Pereira }\\[0.2cm]
    {\Large Gustavo Kletelinger }\\[0.2cm]
    {\Large Matheus Bertoldo de Oliveira }\\[0.4cm]
    \textcolor{cpsgray}{Centro Paula Souza}\\[0.2cm]
    \textcolor{cpsgray}{FATEC Registro}
\end{frame}


% -------------------------------------------------
% SEÇÕES E SLIDES
% -------------------------------------------------

% Logomarca ============================
\section{Logomarca}
\begin{frame}{Logomarca}
\justifying

\begin{figure}[h]
\includegraphics[width=0.8\textwidth]{images/logomarca.png}
\caption{Logomarca}
\label{fig:figure1}
\end{figure}

As formas arredondadas lembram pontos conectados, fazendo referência às redes neurais, 
estruturas fundamentais para o projeto. O laranja é amigável e traz confiança, enquanto o verde
remete à crescimento, saúde e natureza.

\end{frame}
% Logomarca ============================

% Sumário ============================
\begin{frame}{Agenda}
\tableofcontents
\end{frame}
% Sumário ============================

% Pitch ============================
\section{Pitch}
\begin{frame}{Pitch}
\centering
  Assista ao nosso pitch {\color{orange} \href{https://youtu.be/QucX8fwVIDI?si=mnkalk9atQx4rimm}{aqui}}.
\end{frame}
% Pitch ============================

% Problematização 01 ============================
\section{Problematização}
\begin{frame}{Problematização}
\justifying

\begin{wrapfigure}{l}{0.5\textwidth}
\vspace{-30pt}
\includegraphics[width=1.0\linewidth]{images/usda-suco-de-laranja.png} 
\caption{ \href{https://apps.fas.usda.gov/psdonline/circulars/citrus.pdf}{United States Department of Agriculture (2025)}}
\label{fig:wrapfig}
\end{wrapfigure}

O Brasil é o maior produtor de laranja e suco de laranja do mundo, evidenciando a importância nacional dos citros.

\end{frame}
% Problematização 01 ============================

% Problematização 02 ============================
\begin{frame}{Problematização}
\justifying

\begin{figure}[h!]
\centering

O Vale do Ribeira produziu, em 2020, 98 mil toneladas de citros que renderam cerca de 163,2 milhões de reais brutos.

\begin{subfigure}[b]{0.35\textwidth}
\centering
\includegraphics[width=\textwidth]{images/seab-producao-de-citros.png}
%\caption{Imagem 1}
\end{subfigure}
\hfill
\begin{subfigure}[b]{0.35\textwidth}
\includegraphics[width=\textwidth]{images/seab-producao-de-tangerinas.png}
%\caption{Imagem 2}
\end{subfigure}

\caption{ { \href{https://www.agricultura.pr.gov.br/sites/default/arquivos_restritos/files/documento/2021-09/vbp_1920_definitivo.pdf}{Secretaria da Agricultura e do Abastecimento (2020)} } }
\end{figure}

\end{frame}
% Problematização 02 ============================

% Problematização 03 ============================
\begin{frame}{Problematização}

\begin{multicols}{2}

\begin{tikzpicture}[node distance=2cm, auto]

\node (A) [draw, rectangle, inner sep=5pt] {Clima tropical úmido};
\node (B) [draw, rectangle, below of=A, node distance=1cm] {Propiciação de doenças fúngicas};
\node (C) [draw, rectangle, below of=B, node distance=1cm] {Pinta Preta (fungos da espécie \emph{citricarpa})};
\node (D) [draw, rectangle, below of=C, node distance=2cm] 
{
\begin{tabular}{c}
Sintomas  \\
\hline
\colorbox{yellow}{Manchas nos frutos} \\
Necroses foliares \\
Queda prematura \\
\end{tabular}
};
\node (E) [draw, rectangle, below of=D, node distance=2cm]{Prejuízos comerciais};

\draw[->] (A) -- (B);
\draw[->] (B) -- (C);
\draw[->] (C) -- (D);
\draw[->] (D) -- (E);
\end{tikzpicture}

\begin{figure}[h]
\includegraphics[width=0.25\textwidth]{images/sintoma-01.png}
\label{fig:figure2}
\end{figure}

\begin{figure}[h]
\includegraphics[width=0.25\textwidth]{images/sintoma-02.png}
\label{fig:figure3}
\end{figure}

\end{multicols}

\end{frame}
% Problematização 03 ============================

% Problematização 04 ============================
\begin{frame}{Problematização}
\centering

\begin{block}{Remark}
  Sample text
\end{block}

\begin{alertblock}{Remark}
Sample text
\end{alertblock}

\begin{figure}[h]
\includegraphics[width=0.8\textwidth]{images/sintoma-04.png}
\label{fig:figure4}
\end{figure}

\end{frame}
% Problematização 04 ============================


\section{Estado da Arte}
\begin{frame}{Objetivos frágeis}
\justifying
Objetivos gerais pouco operacionais, específicos ausentes ou mal apresentados.  
Há redundâncias e inversão entre o que é produto e o que é método.
\end{frame}

\section{Objetivo}
\begin{frame}{Metodologia confusa ou incompleta}
\justifying
Mistura de etapas, ferramentas e resultados.  
Falta de clareza sobre o fluxo de desenvolvimento, separação entre metodologia de pesquisa e de implementação, e presença de trechos do modelo sem conteúdo autoral.
\end{frame}

\section{Metodologia}
\begin{frame}{Elementos gráficos e estruturais problemáticos}
\justifying
Figuras muito pequenas ou sem legenda.  
Ausência de fluxogramas obrigatórios.  
Tabelas e seções de exemplo do modelo não substituídas adequadamente.
\end{frame}

\section{Apresentação Prática}
\begin{frame}{Citações mal posicionadas ou incompletas}
\justifying
Pedidos de “citação ao final do parágrafo” ignorados.  
Falta de identificação clara da fonte de dados e lacunas entre citações e lista de referências.
\end{frame}


\section{Encerramento}
\begin{frame}{Encerramento}
\justifying
Os problemas levantados indicam a necessidade de fortalecer a formação científica, a escrita acadêmica e o domínio metodológico dos alunos.  
O aprimoramento contínuo desses aspectos é essencial para a qualidade e a credibilidade dos projetos de pesquisa desenvolvidos na FATEC Registro.
\end{frame}

\end{document}
