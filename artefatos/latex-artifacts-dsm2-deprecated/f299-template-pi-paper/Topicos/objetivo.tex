Desenvolver um sistema capaz de identificar sintomas da pinta preta na tangerina ponkan, capturados em foto via câmera de telefone celular, utilizando métodos de Inteligência Artificial e Visão Computacional de modo a otimizar as medidas preventivas em campo. 

Em paralelo, o projeto possui objetivos secundários que complementam sua finalidade principal:\\


\begin{itemize}
    \item Satisfazer parâmetros de tempo de resposta do pré-diagnóstico menor que 5 minutos;
    \item Desenvolver modelos de visão computacional para processamento de imagem;
    \item Desenvolver modelos de redes neurais para análise e diferenciação de sintomas da pinta preta em relação a outras doenças semelhantes;
    \item Construir interfaces de software simples e intuitivas;
    \item Emitir recomendações de manejo personalizadas para o produtor, baseadas em dados climáticos; 
    \item Utilizar conjuntos de dados organizados e diversificados com sintomas da pinta preta para treinamento e teste em rede neural.
\end{itemize}
